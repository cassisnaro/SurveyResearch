\documentclass{report}
\usepackage[utf8]{inputenc}
\usepackage{hyperref}
\usepackage{cite}

\title{Survey Research}
\author{Victoria Beleuta, Sebastian Biewer, Marion Maugendre, Daniel Naro}
\date{October 2014}

\begin{document}

\maketitle
\tableofcontents

\chapter{Introduction}
Survey Research is a tool for the collection of data. This method is used to know the opinions or feelings, in order to analyse, evaluat professional oraganisations, advertasing, ect. It is also used to prove an hypothesis, create hypothesis on a given subjet or identify trends.

Before to begin a survey, some decision must be made and they can have important consequences on the study. There are three areas involved in writting a question. In a first time, the question content, scope and purpose have to be determined. Then, the people need to choose the reponse format that it will be use for collecting information. And the third step is to fugure out how to word the question to get at the issue of interest. 

It exists two categories of surveys : the questionnaire and the interview. The questionnaire can be send to a wide number of people and they allow the respondent to fil lit out at their own convenience. But most of the time, the question are general and the reponses are not detailed. The interview are more personnal. It exits the personnal interview, where the interviewer works directly with the respondent, and the telephone interview.

Once the questionnaires are completed, the responses are analysed and can be publishing as for example on the Worl-Wide Web \cite{intro}

In this paper, we will present you in a first time the design of the survey. Then, we will talk about the survey channels. And at the end we will give you some results analysis and some examples of the use of survey research.
\chapter{Design of a survey}
Each research has certain characteristic and aims, and this should be reflected in the way the research is conducted. The resources provided are limited, and we have to design carefully in order to respect these restrictions. The constraint can be diverse: size of the population, the maximal time that each individual is willing to spend for the survey, or the difficulty to maintain the contact with the individuals over a long lasting survey.

If our study requires us to see the evolution over time for concrete individuals, we will follow the longitudinal study method. We sample a population, and survey them repeatedly over a long period of time. Of course, problems arise in how to maintain contact over time with those we have sampled. The questionee might be willing to leave the survey after a period. If more than one want to leave, the surveyed population we have at the end might not be comparable to the initial one anymore.
We can try to prevent this situation by finding the results of the survey elsewhere. As explained in \cite{JohnShaughnessyEugeneZechmeister2011}, we can use archived data, e.g. medical records over years, as the information. We can also try to complement archived data with new information in order to gain new insights (\cite{Friedman} and \cite{Tucker}). In this studies, the authors studied how the death of a young child could affect the marriage of the parents leading to a divorce.
The main point of a longitudinal research is that we can track the evolution of individuals over time, compared to studies where the surveyed population keep changing.

This kind of surveys are called Successive Independent Samples Studies. They share with the longitudinal study the fact that the survey is repeated over time, but the population sampling keep changing. Most persons have heard about the PISA study: every three years, there is a study conducted among 15 years old children, to assess their level in different fields. The PISA study enables us to study the evolution of the 15 years students' skills over the years, but we cannot say anything about the evolution of those who had previously participated. Moreover we have to be specially careful with the sampling. For example, if one country focuses on schools known to be better than the average, and all others do a unbiased sampling, then the results cannot be compared.  Or if one country changes the way the sampling is done from one edition to the other, than the measured evolution cannot be related to decisions in politics, because we added more factors. Of course, the questions which are asked should be similar or the same too, other way we have yet another way to add reasons for differences in the results.
Successive Independent Samples Studies are helpful when we want to assess the evolution of one aspect of the population over time. This aspect might be an opinion about the politics for example. We are not able to conclude anything about the evolution of each individual, but we can measure global trends.

Finally we can measure just at one point characteristics about the population, by surveying individuals just once. This design, called Cross-Sectional Design, might be useful to study the reaction to an event. For example, Java 8 was released in the first quart of 2014, and Typesafe published the result of a survey about the adoption of this new version in October of the same year. This study informs us about how more than 3000 individuals reacted to the release. In this case the results could help to understand if this release was a success or not, compared to other ones. Another example can be find in \cite{DBLP:dblp_conf/er/TortOP13}, in this case we want to quantify the opinion of former students about one course. More precisely how relevant were different elements of the course for their professional life.

Until now we have focused on surveys which try to assess one characteristic of a population. There is another possible use for surveys: we can use them as means to test an hypothesis.
For example, the study described in \cite{DBLP:dblp_conf/er/TortOP13} could be expanded by surveying another group of former students, who did not take any similar course. In that case we could be studying the hypothesis that having take the course leads to a change in opinion about the course syllabus. This form of testing a hypothesis is  called “Concurrent control studies in which participants are not randomly assigned to group” in \cite{Kitchenham2002}. In this case it is not randomly assigned because the assignation into groups was already given.
If our hypothesis do not require that something has already happen in the past, but in fact we can decide how to split individuals into groups because the hypothesis depends only on events happening during the study, we could reach better initial conditions. For example, in our modified example of the study \cite{DBLP:dblp_conf/er/TortOP13}, we could be in a situation where the group without the course, shared additional differences. Maybe those in this group did not take the course yet because there had not finished their studies, and the results of our study could be in fact depending on this fact rather than on the aspect lecture/ not lecture. If we have the possibility of better controlling the split into two groups, we can make sure that our conclusions depend just on one factor.
To better assess the initial state of each individual, we can complement the survey at the end of the experiment with antoher survey at the beginning. In this case, and because everybody will take both surveys we see similarities with the longitudinal method. The authors of \cite{Kitchenham2002} call this approach “Self-control studies”. As we saw for the longitudinal method, we can use past studies instead of the pre-survey. The authors refer to this as “Historical studies control”.

We have seen different designs that enables us to assess properties of populations, or to test hypothesis by analyzing the evolution of the answers. According to the factors we have to take into account to make the study trustworthy, we can require less constraints: do we need to evaluate exactly the same individuals?, do we need more than one iteration of the survey? If we require this, we will need more resources, maybe resources that we do not have. In the end, we see that there is no one design which fits all survey research. Each design has pro and cons which have to be understood in order to find the best for that study.

\chapter{Survey Channels}
(TODO: Add sources and reread)

In order to carry out a survey, there are several possible channels over which participants can be interviewed. One of the earliest channels was the personal interview. With the broadening of the telephone, personal interviews were more and more replaced by interviews over the phone. Today, we also have the possibility to use the web in order to carry out a survey completely digitally. The mentioned channels are the most important and most used ones, so we explain them in the following sections.

Depending on the purpose of the survey, one channel may be more appropriate than the others. In order to allow to decide whether there is a best channel for a survey, we mention the strength and weaknesses for each channel. Basically, all channels differ in three main aspects aspects. First, there is a huge variety of costs between the presented channels. Second is the degree of influence that an interviewer has on the participant. Depending on the channel, this influence can vary a lot. And last, there can be huge differences regarding the extent of the survey.

All channels have in common, that they can be combined with computer assistance. This allows some additional helpful opportunities for the survey, which we present in section “Computer Assistance”.


\section{Personal Interview}

The oldest channel we present is the personal interview. During a personal interview, the participant and interviewer meet physically. The interviewer has a questionnaire from which he reads questions to the participant who can then give an answer which the interviewer writes down. The location of the interview can be different. It is possible to meet participants in there home or to ask them to participate on the street or a mall.

The strongest advantage of a personal interview is the flexibility it gives both the interviewer and participant. It is possible at any time to ask if something is unclear what the interviewer or participant said. This allows that the participants fully understand the questions they are asked and that the interviewer can write down the participants’ answers correctly. Moreover, the observations the interviewer writes down, may go further than only the words the participant says. Due to the physical presence of the interviewer he knows how convinced the participant is and he can observe the gestures and facial expressions. There are similar possibilities in the other direction as well. In addition to the interviewer just reading questions to the participant, he can use visual elements like advertisements or different variations of a product. The high flexibility during the interview allows a mixture of direct and indirect questions and also to ask about more complex questions than it would be possible in a phone interview or a web survey.

The physical presence of the interviewer has some more advantages. Since he decides where the survey takes place, he has full control over the location. Sometimes, it may be wished that the survey is carried out at a specific location, for example a neutral place such as a shopping mall. The location also can determine the degree of concentration the participant uses to answer the questions. During a phone interview when the TV is running, a participant may be less concentrated than in the environment which is controlled by the interviewer. Another advantage of the physical presence of the interviewer is a higher respond rate of the participants. Due to the face to face communication, people reject an interview less often than for example when they are asked for at the phone or on a web page.

In contrast to the mentioned advantages are some serious problems. In every survey, participants should answer honestly and say their own opinion. However, the presence of the interviewer may already have an influence on the given answers, because the participant may not want to embarrass himself. Also the way how the interviewer asks a question can influence the participant. He might get aware of the expectations the interviewer has about the question which can influence his answer. The influence of the interviewer may also play a role even before the interview starts. When he decides who of the passers-by he asks to participate, he might make the decision based on his personal sympathy with the people.

The second problem that comes with the personal interview channel are its costs. Of all the presented channels, the personal interview is the most expensive one. In order to reduce the interviewer influence mentioned above, they have to organise a special training program for all interviewers. In addition to that, there may be travel costs in order to reach the final location where the survey should take place. The personal interview also needs a lot of organisation, such as deciding how to distribute the interviewers and maybe negotiation with the mall or similar.

\section{Phone Interview}

A big issue on both the pro and contra side of the personal interview is the physical presence of the interviewers. After some time, the personal interviews were replaced more often by phone interviews, which reduced the presence of the interviewers a lot and which also reduced the total costs of the survey.

In order to pick participants, a method called Random-Digit-Dialing (RDD) can be used. RDD just tries phone numbers randomly and if the number exists it plugs in an interviewer. This is more reasonable like using a phone book for example, because not every landline number and especially rarely mobile numbers are listed there. RDD also solves the influence problem that occurs with picking participants for the personal interview.

The reduced presence of the interviewer makes it also more convenient for them to carry out the survey. All interviewers are at the same location from where they make the phone calls. At the same location may take place the interviewer training which prepares the employees to be good interviewers. It is also possible for supervisors to control the work of the interviewers and to guarantee the quality of the survey. They can plug directly into the calls and can here the conversation of participant and interviewer. If he detects mistakes from the interviewer, he can intervene at an early stage.

On the other hand, interviewers have less control in phone interviews. They can not control the environment in which the participant answers the questions, so it might be possible for example, that during the interviewer the participant is distracted by the TV or similar. This may cause that answers are not as deliberate as they could be, so questionnaires should be short and simple in order to get meaningful results. The loss of control is induced by the missing physical presence of the interviewer. This also forbids the interviewer to use more channels during the interview, for example to show advertisements or let the participant use a product. He also can only consider the verbal reactions of the user, but not his gestures or facial expressions. In phone surveys, the interviewer also looses the above mentioned channels for asking people to participate. It is much simpler to get rid of the interviewer at the phone than it is when he asks personally. Due to the loss of communication channels, the inference of the interviewer to the participant is less than in the personal interview, however there can still be some kind of influence, for example caused by how the interviewer asks the questions.

Usually, people prefer to take surveys only if they know that their privacy is protected and their answers remain anonymous. In a personal interview when the participant is picked randomly, this anonymity is pretty much ensured as long as the participant does not reveal any personal information from which his identity can be concluded. When the interviewer calls someone for an interview, he obviously has his phone number, which can identify the participant very well, because that is how he was specifically addressed. Of course, survey systems hide the dialled phone number and does not connect it to the answered questionnaire, but it is difficult to convince the potential participant about this.

\section{Web Survey}

The activity of the interviewer already decreased from the personal interview to the pone interview. The final step that can be made in this aspect, is to remove the interviewer completely. This is done in web surveys, which are carried out by the participant on his own and of course the web server which provides the questionnaire. Depending on the type of the survey, it can be designed to be either public, so everyone who wants can participate or to be protected, in order to have control about who can participate.

The capability of precisely defining and controlling the target of the survey is one of the strength of web surveys. It is possible to do the selection online or offline. We do not want to go into details how offline selection may work, however once a set of people is chosen offline, each of them may get its own credentials to participate in the survey. For an online selection it is for example possible to present a screening questionnaire first and depending on the results, the participant will be granted access to the survey or not. Another selection method may be to allow only certain IP addresses from a certain country or to allow only users of Twitter with a minimum number of followers.

There can be very strict selection criteria, because the size of the accessible population in the internet is in general very large. This even allows to get results from a group of people with a very narrow topic domain. Despite these powerful control mechanisms the costs for web surveys can be very low in comparison to phone or even personal surveys. The initiator of the survey can create the questionnaire with a special tool, which can then provide it online. It is not necessary to print the questionnaires, the answers can be automatically provided by the tool in an appropriate digital format and it is not necessary to hire interviewers. The absence of interviewers also ensure that the participants are influenced by them as it was the problem in the survey channels described above.

One of the very large strengths of web surveys is their control over the group of participants. However, this degree of control decreases in the actual answering process. People may not answer questions honestly, choose answers randomly or make a fun of choosing the funniest answers. It can also not be ensured that people participate only once, because people of the target group will also be in the target group when they try to participate again, so they will be granted access again. Moreover it is possible that participants do not understand the question, but there is no interviewer who they can call back to in order to clarify. As already mentioned in the phone survey section, it is also not possible to consider the gestures and facial expressions of the participants, because the web system can only respect the answers given in the web form.

Narrowing down the target of participants is often simple for web surveys. The other way around however is difficult or even impossible. Making the sample as general as possible requires that people of all ages use the internet and can be reached to advertise the survey. This is however not the case, because especially older people are very underrepresented in the web.

To conclude, web surveys are very powerful and cheap for certain topics. However it may also be that it is even possible to get meaningful results, especially when people of all ages should be interviewed.

\section{Computer Assistance}

The Web Surveys from the previous has the powerful advantage of being served by a computer. This allows a lot of interventions during the course of the survey. It is for example possible to show the questions in a random order to ensure that the order of the question has no impact on the given answers. The order of the questions can be determined more strategic as well. Depending on the answers for some questions the system decided which questions should be displayed next. It may even be possible to not show some questions at all depending in the already collected answers.

These features are also available for phone (Computer-Assisted-Telephone-Interview) and personal interviews (Computer-Assisted-Personal-Interview). The interview can use a computer which shows the questions and where he writes the answers of the participant. In personal interviews however it is necessary to ensure that the batteries of the computer last long enough. In phone interviews the role of the computer may be even extended. Instead of the interviewer reading each questions again and again in each interview, it is possible to record the interviewer asking the question and then just playing the recordings in the actual interview. This ensures that the question is asked the same way every time and relieves the interviewer.




\chapter{Results analysis}
In order to have meaningful data to analyse, we want to have enough participants for our study. However, it is often very hard to motivate people to take our survey and to provide us with accurate and complete answers. Survey researchers often use small monetary rewards or gifts, but this often proves ineffective. In most casses, people will complete a survey if they know how this study will benefit them. For this reason it is important to provide the participants with the following information\cite{Kitchenham4}:
\begin{itemize}
\item the purpose of the study;\\
\item relevance to the participant;\\
\item importance of each participant's response;\\
\item the process of selecting the participants;\\
\item how the confidentiality of the participants is preserved;\\
\end{itemize}

After all the responses have been collected, we can start analysing our data. The first step is the data validation, meaning all responses have to be checked for completeness and consistency. If our initial design of the study was well thought out, the incomplete responses should be a small part of the total data. However, if our questions are ambiguous, we can have a lot of erroneous data or blanks. In order to avoid these problems, it is recommended to use pilot surveys during our design phase. Pilot surveys help us identify which questions are most likely to cause unusuable data in the future and we can change them accordingly. 

However, even if we take all the necessary steps in designing our study, different types of incomplete responses occur often, and there are different methods we can use to analyse them. Blanks can be ignored from our sample, but only after we have a look at the questions left blank to make sure there is no systematic bias introduced \cite{Kitchenham6}. If we notice that one or two particular questions have been left blank by more than one participant, we can chose to ignore only that question for all responses. If more than a couple of questions have not been answered by different groups of people, it is more appropriate to analyse all of the responses and present the results for each question separately, keeping in mind to present the different sample size for each of them. Lastly, if we only have a few blanks from a couple of participants, we can simply chose to ignore those responses.

After validating the data and deciding on how we want to present it, we go on to data coding. The types of results that usually are gathered in surveys are:
\begin{itemize}
\item dichotomous - participants are given two choices;\\
\item nominal - participants are given more than two unordered options;\\
\item ordinal - participants are given more than two ordered options;\\
\item continuous;
\end{itemize} 
In order to use many of the statistical packages available, we have to convert nominal and ordinal scale data from category names to numerical scores. Often, the coding of data is done during the design phase, however open ended questions require human expertise to decide whether two results are equivalent or not. 

There are a large number of surveys that ask people to answer questions based on a 5 point agreement scale, which is one type of \textit{ordinal scale}. It is easy to convert the results into the numerical equivalent and analyze them. However, in some cases, this approach can violate the mathematical rules for analyzing ordinal data. For example, if the data has no central tendency, the mean has no value in correlation with the actual results. This can be the case for a bimodal distribution which tells us that a large number of participants "strongly disagree" and a large number of participants "strongly agree", and the mean value could imply the wrong conclusion that most people "neighter agree or disagree". As we can see, ordinal data is very easy to collect and provides us with a lot of information but we have to make sure we look at the "big picture", such that we can use the statistical packets correctly. The most common form of analysis applied to \textit{nominal data} is to determine the proportion of responses in each category, but we can also use multi-way tables and chi-squared tests to measure different associations \cite{Kitchenham6}.

\chapter{Example}
In this section we will look at the paper "A survey of the relevance of computer science and software engineering education" conducted by Timothy C. Lethbridge in 1998. In this study, software developers were asked a series of questions about their education and how relevant it has been to their present careers. The collected data aims at helping educational institutions and companies to create a better curricula and training programs. The survey consisted of a series of demographic questions, followed by a larger set of questions related to topics in mathematics, software and engineering. For each topic participants were asked four key questions, which were answered on a 0 ("learned nothing at all") to 5("learned in depth") scale: 
\begin{itemize}
\item How much did you learn about this topic during your education?\\
\item What is your current knowledge of this topic?\\
\item How useful has this knowledge been in your career?\\
\item How useful for your career would it have been to learn more about this topic?\\
\end{itemize}

The sampling process consisted of approaching companies and asking to have the paper-based or internet-based survey to their employees. While some management teams declined the request, six companies were enthusiastic to help with the study. The study was successful in gathering responses from people with diverse background education, years of expertise and careers, however, the paper also underlines some possible biases such as location (most participants belonged to the North American region), postgraduate degrees (33\% of participants had a graduate education which is higher than the general proportion in software professions), and others.

Out of all the participants 7\% had at most college education, 60\% had at most a bachelor's degree, 27\% had at most a master's degree and 6\% reported having obtained a Phd degree, with an average of 9.6 years since their last degree. Analyzing the general relevance of their education to their career, the results showed a mean of 3.5 (described as "relevant at times"). However, 51\% considered their education "very relevant" while the rest were not so sure. In order to further analyse the results, the responses were split into two groups: those who answeres with $\geq4$ and $\prec4$.  Respondents educated in the US were significantly more likely to find their education
relevant (65\%), and those with software engineering education, as expected, considered it more relevant than those with a different background. 

One of the interesting observations was the fact that junior respondents found their education less relevant (43\%) in comparison to their senior coworkers. The researchers provide two possible explanations for this: "1) more senior people might have had the chance to work on a wider variety of projects which, taken cumulatively, make use of a greater fraction of their education, or 2) education might be becoming less relevant. We prefer the first explanation." \cite{lethbridge}

To sum up, the study definitely provided an interesting peek into the correlation between the software engineering education and the corresponding positions. However, we should keep in mind that this survey was conducted in 1998 and we can safely assume that nowadays the situation has changed, due to a few factors such as: open source projects, new career paths, extensive amount of information available outside educational institutions, etc.

\chapter{Conclusion}
The survey research is a way to collect usefull data thanks some channels like questionnaires or personnal interview. But the first step of your research is to know what you want to study and how, it's mean over a long period of time, as Successive Independent Samples Studies, or just once, as Cross-Sectional Design, or assess one characteristic of the population or test an hypothesis.
Once you have decided which king of design survey you want, the channels survey are also important cause it can be very personnal, as personnal interview, or more impersonnal as web survey or computer assitance because there is not physical contact with the interviewer.
At the end, the last step is to analyse the data in order to have statistics which can be usefull, for example to help educational institutions and companies to improve themself.

\bibliography{mybib}{}
\bibliographystyle{unsrt}
\end{document}

