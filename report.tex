\documentclass{article}
\usepackage[utf8]{inputenc}
\usepackage{cite}

\title{Survey Research}
\author{blabla}
\date{October 2014}

\begin{document}

\maketitle

\section{Introduction}
\section{Design of a survey}
Each research has certain characteristic and aims, and this should be reflected in the way the research is conducted. The resources provided are limited, and we have to design carefully in order to respect these restrictions. The constraint can be diverse: size of the population, the maximal time that each individual is willing to spend for the survey, or the difficulty to maintain the contact with the individuals over a long lasting survey.

If our study requires us to see the evolution over time for concrete individuals, we will follow the longitudinal study method. We sample a population, and survey them repeatedly over a long period of time. Of course, problems arise in how to maintain contact over time with those we have sampled. The questionee might be willing to leave the survey after a period. If more than one want to leave, the surveyed population we have at the end might not be comparable to the initial one anymore.
We can try to prevent this situation by finding the results of the survey elsewhere. As explained in \cite{JohnShaughnessyEugeneZechmeister2011}, we can use archived data, e.g. medical records over years, as the information. We can also try to complement archived data with new information in order to gain new insights (\cite{Friedman} and \cite{Tucker}). In this studies, the authors studied how the death of a young child could affect the marriage of the parents leading to a divorce.
The main point of a longitudinal research is that we can track the evolution of individuals over time, compared to studies where the surveyed population keep changing.

This kind of surveys are called Successive Independent Samples Studies. They share with the longitudinal study the fact that the survey is repeated over time, but the population sampling keep changing. Most persons have heard about the PISA study: every three years, there is a study conducted among 15 years old children, to assess their level in different fields. The PISA study enables us to study the evolution of the 15 years students' skills over the years, but we cannot say anything about the evolution of those who had previously participated. Moreover we have to be specially careful with the sampling. For example, if one country focuses on schools known to be better than the average, and all others do a unbiased sampling, then the results cannot be compared.  Or if one country changes the way the sampling is done from one edition to the other, than the measured evolution cannot be related to decisions in politics, because we added more factors. Of course, the questions which are asked should be similar or the same too, other way we have yet another way to add reasons for differences in the results.
Successive Independent Samples Studies are helpful when we want to assess the evolution of one aspect of the population over time. This aspect might be an opinion about the politics for example. We are not able to conclude anything about the evolution of each individual, but we can measure global trends.

Finally we can measure just at one point characteristics about the population, by surveying individuals just once. This design, called Cross-Sectional Design, might be useful to study the reaction to an event. For example, Java 8 was released in the first quart of 2014, and Typesafe published the result of a survey about the adoption of this new version in October of the same year. This study informs us about how more than 3000 individuals reacted to the release. In this case the results could help to understand if this release was a success or not, compared to other ones. Another example can be find in \cite{DBLP:dblp_conf/er/TortOP13}, in this case we want to quantify the opinion of former students about one course. More precisely how relevant were different elements of the course for their professional life.

Until now we have focused on surveys which try to assess one characteristic of a population. There is another possible use for surveys: we can use them as means to test an hypothesis.
For example, the study described in \cite{DBLP:dblp_conf/er/TortOP13} could be expanded by surveying another group of former students, who did not take any similar course. In that case we could be studying the hypothesis that having take the course leads to a change in opinion about the course syllabus. This form of testing a hypothesis is  called “Concurrent control studies in which participants are not randomly assigned to group” in \cite{Kitchenham2002}. In this case it is not randomly assigned because the assignation into groups was already given.
If our hypothesis do not require that something has already happen in the past, but in fact we can decide how to split individuals into groups because the hypothesis depends only on events happening during the study, we could reach better initial conditions. For example, in our modified example of the study \cite{DBLP:dblp_conf/er/TortOP13}, we could be in a situation where the group without the course, shared additional differences. Maybe those in this group did not take the course yet because there had not finished their studies, and the results of our study could be in fact depending on this fact rather than on the aspect lecture/ not lecture. If we have the possibility of better controlling the split into two groups, we can make sure that our conclusions depend just on one factor.
To better assess the initial state of each individual, we can complement the survey at the end of the experiment with antoher survey at the beginning. In this case, and because everybody will take both surveys we see similarities with the longitudinal method. The authors of \cite{Kitchenham2002} call this approach “Self-control studies”. As we saw for the longitudinal method, we can use past studies instead of the pre-survey. The authors refer to this as “Historical studies control”.

We have seen different designs that enables us to assess properties of populations, or to test hypothesis by analyzing the evolution of the answers. According to the factors we have to take into account to make the study trustworthy, we can require less constraints: do we need to evaluate exactly the same individuals?, do we need more than one iteration of the survey? If we require this, we will need more resources, maybe resources that we do not have. In the end, we see that there is no one design which fits all survey research. Each design has pro and cons which have to be understood in order to find the best for that study.

\bibliography{mybib}{}
\bibliographystyle{plain}
\end{document}

