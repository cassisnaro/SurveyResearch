\documentclass{article}
\usepackage[utf8]{inputenc}
\usepackage{cite}

\title{Survey Research}
\author{Victoria Bella, Sebastian Biewer, Marion Maugendre, Daniel Naro}
\date{October 2014}

\begin{document}

\maketitle

\section{Introduction}
\section{Design of a survey}
Each research has certain characteristic and aims, and this should be reflected in the way the research is conducted. The resources provided are limited, and we have to design carefully in order to respect these restrictions. The constraint can be diverse: size of the population, the maximal time that each individual is willing to spend for the survey, or the difficulty to maintain the contact with the individuals over a long lasting survey.

If our study requires us to see the evolution over time for concrete individuals, we will follow the longitudinal study method. We sample a population, and survey them repeatedly over a long period of time. Of course, problems arise in how to maintain contact over time with those we have sampled. The questionee might be willing to leave the survey after a period. If more than one want to leave, the surveyed population we have at the end might not be comparable to the initial one anymore.
We can try to prevent this situation by finding the results of the survey elsewhere. As explained in \cite{JohnShaughnessyEugeneZechmeister2011}, we can use archived data, e.g. medical records over years, as the information. We can also try to complement archived data with new information in order to gain new insights (\cite{Friedman} and \cite{Tucker}). In this studies, the authors studied how the death of a young child could affect the marriage of the parents leading to a divorce.
The main point of a longitudinal research is that we can track the evolution of individuals over time, compared to studies where the surveyed population keep changing.

This kind of surveys are called Successive Independent Samples Studies. They share with the longitudinal study the fact that the survey is repeated over time, but the population sampling keep changing. Most persons have heard about the PISA study: every three years, there is a study conducted among 15 years old children, to assess their level in different fields. The PISA study enables us to study the evolution of the 15 years students' skills over the years, but we cannot say anything about the evolution of those who had previously participated. Moreover we have to be specially careful with the sampling. For example, if one country focuses on schools known to be better than the average, and all others do a unbiased sampling, then the results cannot be compared.  Or if one country changes the way the sampling is done from one edition to the other, than the measured evolution cannot be related to decisions in politics, because we added more factors. Of course, the questions which are asked should be similar or the same too, other way we have yet another way to add reasons for differences in the results.
Successive Independent Samples Studies are helpful when we want to assess the evolution of one aspect of the population over time. This aspect might be an opinion about the politics for example. We are not able to conclude anything about the evolution of each individual, but we can measure global trends.

Finally we can measure just at one point characteristics about the population, by surveying individuals just once. This design, called Cross-Sectional Design, might be useful to study the reaction to an event. For example, Java 8 was released in the first quart of 2014, and Typesafe published the result of a survey about the adoption of this new version in October of the same year. This study informs us about how more than 3000 individuals reacted to the release. In this case the results could help to understand if this release was a success or not, compared to other ones. Another example can be find in \cite{DBLP:dblp_conf/er/TortOP13}, in this case we want to quantify the opinion of former students about one course. More precisely how relevant were different elements of the course for their professional life.

Until now we have focused on surveys which try to assess one characteristic of a population. There is another possible use for surveys: we can use them as means to test an hypothesis.
For example, the study described in \cite{DBLP:dblp_conf/er/TortOP13} could be expanded by surveying another group of former students, who did not take any similar course. In that case we could be studying the hypothesis that having take the course leads to a change in opinion about the course syllabus. This form of testing a hypothesis is  called “Concurrent control studies in which participants are not randomly assigned to group” in \cite{Kitchenham2002}. In this case it is not randomly assigned because the assignation into groups was already given.
If our hypothesis do not require that something has already happen in the past, but in fact we can decide how to split individuals into groups because the hypothesis depends only on events happening during the study, we could reach better initial conditions. For example, in our modified example of the study \cite{DBLP:dblp_conf/er/TortOP13}, we could be in a situation where the group without the course, shared additional differences. Maybe those in this group did not take the course yet because there had not finished their studies, and the results of our study could be in fact depending on this fact rather than on the aspect lecture/ not lecture. If we have the possibility of better controlling the split into two groups, we can make sure that our conclusions depend just on one factor.
To better assess the initial state of each individual, we can complement the survey at the end of the experiment with antoher survey at the beginning. In this case, and because everybody will take both surveys we see similarities with the longitudinal method. The authors of \cite{Kitchenham2002} call this approach “Self-control studies”. As we saw for the longitudinal method, we can use past studies instead of the pre-survey. The authors refer to this as “Historical studies control”.

We have seen different designs that enables us to assess properties of populations, or to test hypothesis by analyzing the evolution of the answers. According to the factors we have to take into account to make the study trustworthy, we can require less constraints: do we need to evaluate exactly the same individuals?, do we need more than one iteration of the survey? If we require this, we will need more resources, maybe resources that we do not have. In the end, we see that there is no one design which fits all survey research. Each design has pro and cons which have to be understood in order to find the best for that study.


\section{Survey Channels}
In order to carry out a survey, there are several possible channels over which participants can be interviewed. One of the earliest channels was the personal interview. With the broadening of the telephone, personal interviews were more and more replaced by interviews over the phone. Today, we also have the possibility to use the web in order to carry out a survey completely digitally. The mentioned channels are the most important and most used ones, so we explain them in the following sections.

Depending on the purpose of the survey, one channel may be more appropriate than the others. In order to allow to decide whether there is a best channel for a survey, we mention the strength and weaknesses for each channel. Basically, all channels differ in three main aspects aspects. First, there is a huge variety of costs between the presented channels. Second is the degree of influence that an interviewer has on the participant. Depending on the channel, this influence can vary a lot. And last, there can be huge differences regarding the extent of the survey.

All channels have in common, that they can be combined with computer assistance. This allows some additional helpful opportunities for the survey, which we present in section “Computer Assistance”.


\subsection{Personal Interview}

The oldest channel we present is the personal interview. During a personal interview, the participant and interviewer meet physically. The interviewer has a questionnaire from which he reads questions to the participant who can then give an answer which the interviewer writes down. The location of the interview can be different. It is possible to meet participants in there home or to ask them to participate on the street or a mall.

The strongest advantage of a personal interview is the flexibility it gives both the interviewer and participant. It is possible at any time to ask if something is unclear what the interviewer or participant said. This allows that the participants fully understand the questions they are asked and that the interviewer can write down the participants’ answers correctly. Moreover, the observations the interviewer writes down, may go further than only the words the participant says. Due to the physical presence of the interviewer he knows how convinced the participant is and he can observe the gestures and facial expressions. There are similar possibilities in the other direction as well. In addition to the interviewer just reading questions to the participant, he can use visual elements like advertisements or different variations of a product. The high flexibility during the interview allows a mixture of direct and indirect questions and also to ask about more complex questions than it would be possible in a phone interview or a web survey.

The physical presence of the interviewer has some more advantages. Since he decides where the survey takes place, he has full control over the location. Sometimes, it may be wished that the survey is carried out at a specific location, for example a neutral place such as a shopping mall. The location also can determine the degree of concentration the participant uses to answer the questions. During a phone interview when the TV is running, a participant may be less concentrated than in the environment which is controlled by the interviewer. Another advantage of the physical presence of the interviewer is a higher respond rate of the participants. Due to the face to face communication, people reject an interview less often than for example when they are asked for at the phone or on a web page.

In contrast to the mentioned advantages are some serious problems. In every survey, participants should answer honestly and say their own opinion. However, the presence of the interviewer may already have an influence on the given answers, because the participant may not want to embarrass himself. Also the way how the interviewer asks a question can influence the participant. He might get aware of the expectations the interviewer has about the question which can influence his answer. The influence of the interviewer may also play a role even before the interview starts. When he decides who of the passers-by he asks to participate, he might make the decision based on his personal sympathy with the people.

The second problem that comes with the personal interview channel are its costs. Of all the presented channels, the personal interview is the most expensive one. In order to reduce the interviewer influence mentioned above, they have to organise a special training program for all interviewers. In addition to that, there may be travel costs in order to reach the final location where the survey should take place. The personal interview also needs a lot of organisation, such as deciding how to distribute the interviewers and maybe negotiation with the mall or similar.

\subsection{Phone Interview}

A big issue on both the pro and contra side of the personal interview is the physical presence of the interviewers. After some time, the personal interviews were replaced more often by phone interviews, which reduced the presence of the interviewers a lot and which also reduced the total costs of the survey.

In order to pick participants, a method called Random-Digit-Dialing (RDD) can be used. RDD just tries phone numbers randomly and if the number exists it plugs in an interviewer. This is more reasonable like using a phone book for example, because not every landline number and especially rarely mobile numbers are listed there. RDD also solves the influence problem that occurs with picking participants for the personal interview.

The reduced presence of the interviewer makes it also more convenient for them to carry out the survey. All interviewers are at the same location from where they make the phone calls. At the same location may take place the interviewer training which prepares the employees to be good interviewers. It is also possible for supervisors to control the work of the interviewers and to guarantee the quality of the survey. They can plug directly into the calls and can here the conversation of participant and interviewer. If he detects mistakes from the interviewer, he can intervene at an early stage.

TODO: Cons

\subsection{Web Survey}

TODO

\subsection{Computer Assistance}

TODO




\bibliography{mybib}{}
\bibliographystyle{plain}
\end{document}

